\documentclass{article}

% Language setting
% Replace `english' with e.g. `spanish' to change the document language
\usepackage[english]{babel}

% Set page size and margins
% Replace `letterpaper' with`a4paper' for UK/EU standard size
\usepackage[letterpaper,top=2cm,bottom=2cm,left=3cm,right=3cm,marginparwidth=1.75cm]{geometry}

% Useful packages
\usepackage{amsmath}
\usepackage{graphicx}
\usepackage[colorlinks=true, allcolors=blue]{hyperref}

\title{Assignment 1}
\author{Your name}

\begin{document}
\maketitle


\section{Correlation [3 points]}
Consider the dataset in this workbook:
\href{https://docs.google.com/spreadsheets/d/1wMSp2pQDvF64spFXlWUCd9H6N8GVtmsWD3-NTqDV74s/edit?usp=sharing}{link}, in tab \emph{correlation}.

Make a copy of the workbook, and change the sharing settings for everyone to view.\\

Create a scatter plot of data. What do you think the correlation coefficient will be?

Answer: I think the correlation will be in the range between $0.$ and $0.$.\\

Now compute Pearson correlation coefficient using one of the approaches discussed during the Recitation. Explain which approach did you use and show intermediate results in the workbook.

Answer: The correlation coefficient I computed is $r=$. 

\newpage
\section{Frequent items [4 points]}
Suppose there are 100 items, numbered 1 to 100, and also 100
transactions, also numbered 1 to 100. Item $i$ is in transaction $b$ if and only if $i$ divides $b$
with no remainder: $\mod{b,i}=0$. Thus, item 1 is in all the transactions, item 2 is in all fifty of the
even-numbered transactions, and so on. Transaction 12 consists of items {1, 2, 3, 4, 6, 12},
since these are all the integers that are divisors of 12.\\ 

Find the answer to the following question:
If the support threshold is 5 out of 100, which items are frequent?

Give some explanations of how did you arrive to this answer.\\

Arguments:

Answer: 

\newpage
\section{Association Rules [3 points]}
Compute the confidence of the following association rules using the data in tab \emph{association}.
\begin{enumerate}
    \item \textbf{if Tea then Coffee}\\ $conf=\frac{num1}{num2}=num3$
    \item \textbf{if Coffee then Tea}\\ $conf=\frac{num1}{num2}=num3$
    \item \textbf{if Coffee then Not Tea}\\ $conf=\frac{num1}{num2}=num3$
\end{enumerate}

\end{document}



